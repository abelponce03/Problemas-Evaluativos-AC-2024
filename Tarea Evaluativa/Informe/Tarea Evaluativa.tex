%%%%%%%%%%%%%%%%%%%%%%%%%%%%%%%%%%%%%%%%%%%%%%%%%%%%%%%%%%%%%%%%%%%%%%%%%%%
%
% Plantilla para un artículo en LaTeX en español.
%
%%%%%%%%%%%%%%%%%%%%%%%%%%%%%%%%%%%%%%%%%%%%%%%%%%%%%%%%%%%%%%%%%%%%%%%%%%%

% Qué tipo de documento estamos por comenzar:
\documentclass[a4paper]{article}
% Esto es para que el LaTeX sepa que el texto está en español:
\usepackage[spanish]{babel}
\selectlanguage{spanish}
% Esto es para poder escribir acentos directamente:
\usepackage[utf8]{inputenc}
\usepackage[T1]{fontenc}
\usepackage{listings}
\usepackage{amsmath}
\usepackage{graphicx}
\usepackage{color}
\usepackage{xcolor}
\usepackage[utf8]{inputenc}
\usepackage[T1]{fontenc}
\usepackage{listings}
\usepackage{tikz}
\usepackage{listingsutf8}
\usepackage{titlesec}
\usepackage{sectsty}
\usepackage{fontawesome}
\usetikzlibrary{shapes,arrows,positioning}


\lstset{
	language=[x86masm]Assembler,
	inputencoding=utf8,
	language=Python,
	basicstyle=\ttfamily\small,
	commentstyle=\color{green!50!black},
	keywordstyle=\color{blue},
	stringstyle=\color{red},
	numbers=left,
	numberstyle=\tiny\color{gray},
	stepnumber=1,
	numbersep=5pt,
	backgroundcolor=\color{white},
	showspaces=false,
	showstringspaces=false,
	showtabs=false,
	frame=single,
	rulecolor=\color{black},
	tabsize=4,
	captionpos=b,
	breaklines=true,
	breakatwhitespace=false,
	title=\lstname,
	escapeinside={\%*}{*)},
	morekeywords={def,for,in,range,if,else,return},
	}


%% Asigna un tamaño a la hoja y los márgenes
\usepackage[a4paper,top=3cm,bottom=2cm,left=3cm,right=3cm,marginparwidth=1.75cm]{geometry}

%% Paquetes de la AMS
\usepackage{amsmath, amsthm, amsfonts}
%% Para añadir archivos con extensión pdf, jpg, png or tif
\usepackage{graphicx}
\usepackage[colorinlistoftodos]{todonotes}
\usepackage[colorlinks=true, allcolors=blue]{hyperref}

%% Primero escribimos el título
\title{\textbf {Tarea Evaluativa} }
\author{Abel Ponce González C211\\
	Richard Alejandro Matos Arderí C211	
}


%% Después del "preámbulo", podemos empezar el documento

\begin{document}
	\maketitle

	\begin{flushleft}

      		\section*{Ejercicio \textcolor{red}{18}}
      	     \rule{\linewidth}{2pt}	
      		
      		\subsection*{Descripción}
      		
      		Dada una lista $L$ de $n$ elementos $a_{1}, a_{2}, ... a_{n}$ devolver el índice del mínimo elemento de $L$.
      		
      		\subsection*{Salida}
      		
      		Para la salida debe imprimir el mínimo elemento de $L$.
      		
      		Por ejemplo: para $L = [4, 3, 5, 6]$ debería imprimir: 
      		
      		\begin{lstlisting}
      			1
      		\end{lstlisting}      		
      		
      		\subsection*{Logisim}
      		
      		Se dispondrá en INPUT los datos de entrada a partir de la dirección 0. La entrada se estructura de la siguiente forma:
      		
      		\begin{itemize}
      			
      			\item $w_{0}: n $(Tamaño de la lista $L$)
      			\item $w_{1:n} : L$
      		\end{itemize}
      		
      		\subsection*{SASM}
      		
      		En la sección .data se deben definir los valores de entrada de la siguiente forma:
      		
      		\begin{itemize}
      			
      			\item $n$ : un número de tamaño $dd$ que representa al tamaño de la lista $L$
      			
      			\item $array$ : un array de números de tamaño $dd$ que representa $L$
      			
      			
      		\end{itemize}
      	
      		Por ejemplo, un posible encabezado podría ser:
      		
      		\begin{lstlisting}
      			section .data
      			n dd 4
      			array dd 4, 3, 5, 6
      		\end{lstlisting}
      		
      		
      		\subsection*{Seudocódigo}
      		
      	      \begin{lstlisting} [language=Python]
      	     def Minimun(size, lista):
      	     	a = 0
      	     	i = 1
      	     		while i < size:
      	     			if lista[i] < lista[a]:
      	     				a = i
      	     			i = i + 1
      	     	return a 
      	     \end{lstlisting}
      		 
      		 \begin{figure}[ht]
      		 	\centering
      		 	\subsection*{DIAGRAMA}
      		 	\begin{tikzpicture}
      		 		
      		 		\node (idle) [fill= blue, text= white, rectangle] {IDLE};
      		 		\node (start) [fill= red, diamond, draw = orange, below= 0.3cm of idle] {START};
      		 		\node (asignar) [fill=green, circle, draw=yellow, below=0.3cm of start, align=center] {$n = w_0$\\$a = 0$ \\ $i = 1$ \\ $min = L[0]$};
      		 		\node (while) [fill=blue, text= white, rectangle, below= 0.3 of asignar] {WHILE};
      		 		\node (i < n) [fill= red, diamond, draw = orange, below= 0.3cm of while] {i < n};
      		 		\node (end) [fill= blue, text= white, rectangle, left = 0.3cm of i < n] {END};
      		 		
      		 		\node (l[i] < l[a]) [fill= red, diamond, draw = orange, below= 0.3cm of i < n] {L[i] < L[a]};
      		 		
      		 	\node (update) [fill=green, circle, draw=yellow, below =0.3cm of {l[i] < l[a]}, align=center] {$a = i$};
      		 	
      		 		\node (after if) [fill=blue, text= white, rectangle, right= 0.3 of i < n] {AFTER IF};
      		 		
      		 		\node (incremento) [fill=green, circle, draw=yellow, right=0.47cm of while, align=center] {$i = i + 1$};
      		 		
      		 		
      		 		\draw[->,thick] (idle) -- (start);
      		 		\draw[->,thick] (start) -- (asignar);
      		 		\draw[->,thick] (asignar) -- (while);
      		 		\draw[->,thick] (start) -- (2,-1.5) -- (2,0) -- (idle);
      		 	    \draw[->,thick] (while) -- (i < n);
      		 	    \draw[->,thick] (i < n) -- (end);
      		 	    \draw[->,thick] (end) -- (-1.55,0) -- (idle);
      		 	    \draw[->,thick] (i < n) -- (l[i] < l[a]);
      		 	    \draw[->,thick] (l[i] < l[a]) -- (update);
      		 	    \draw[->,thick] (incremento) -- (while);
      		 	    \draw[->,thick] (after if) -- (incremento);
      		 	    \draw[->,thick] (l[i] < l[a]) -- (1.5, -9.49) -- (1.5, -7.43);
      		 	    \draw[->,thick] (update) -- (2.6,-11.57) -- (2.6, - 7.43);
      		 		
      		 	\end{tikzpicture}
      		 \end{figure}
      		
      		\section*{Ejercicio \textcolor{red}{62}}
      		 \rule{\linewidth}{2pt}	
      		\subsection*{Descripción}
      		
      		 Determinar si una lista $a_{1}, a_{2}, ... a_{i}$ está ordenada en orden creciente o decreciente.
      		 
      		 \subsection*{Salida}
      		 
      		 Para la salida debe imprimir la C si está ordenada en orden creciente o D lo está en orden decreciente.\newline
      		 
      		 Por ejemplo: para $L = [3, 4, 6]$ debería imprimir:
      		 
      		 \begin{lstlisting}
      		 	C
      		 \end{lstlisting}
      		 
      		\subsection*{Logisim}
      		
      		Se dispondrá en INPUT los datos de entrada a partir de la dirección 0. La entrada se estructura de la siguiente forma:
      		
      		\begin{itemize}
      			
      			\item $w_{0}: n $(Tamaño de la lista $L$)
      			\item $w_{1:n} : L$
      		\end{itemize}
      		 
      		 	\subsection*{SASM}
      		 
      		 En la sección .data se deben definir los valores de entrada de la siguiente forma:
      		 
      		 \begin{itemize}
      		 	
      		 	\item $n$ : un número de tamaño $dd$ que representa al tamaño de la lista $L$
      		 	
      		 	\item $array$ : un array de números de tamaño $dd$ que representa $L$
      		 	
      		 	
      		 \end{itemize}
      		 
      		 Por ejemplo, un posible encabezado podría ser:
      		 
      		 \begin{lstlisting}
      		 	section .data
      		 	n dd 3
      		 	array dd 3, 4, 6
      		 \end{lstlisting}
      		 
      		 \subsection*{Seudocódigo}
      		 
      		 \begin{lstlisting} [language=Python]
      		 	def Ordenacion (n, lista):
      		 	
      		 		C = False
      		 		D = False
      		 		i = 0
      		 	
      		 		while i < n - 1:
      		 	
      		 			if lista[i + 1] > lista[i]:  C = True
      		 	
      		 			elif lista[i + 1] < lista[i]:  D = True
      		 	
      		 			if D & C : break
      		 	
      		 			i = i + 1
      		 	
      		 		if D & C : return "NO ORDENADA"
      		 		elif D : return "D"
      		 		else : return "C"
      		 \end{lstlisting}
      		  \begin{figure}[ht]
      		 	\centering
      		 	\subsection*{DIAGRAMA}
      		 	\begin{tikzpicture}
      		 		
      		 		\node (idle) [fill= black, text= white, rectangle] {IDLE};
      		 		\node (start) [fill= purple, text=white, diamond, draw = red, below= 0.3cm of idle] {START};
      		 		\node (asignar) [fill=pink, circle, below=0.3cm of start, align=center] {$n= w_0$\\$C = 0$ \\ $D = 0$ \\ $i = 0$};
      		 		
      		 		\node (while) [fill=black, text= white, rectangle, below= 0.3 of asignar] {WHILE};
      		 		\node (i < n - 1) [fill= purple, text=white, diamond, draw = red, below= 0.5cm of while] {i < n - 1};
      		 		\node (end) [fill= black, text= white, rectangle, left = 0.3cm of i < n - 1] {END};
      		 		
      		 		\node (l[i + 1] > l[i]) [fill= purple, text= white, diamond, draw = red, below= 0.4cm of i < n] {L[i + 1] > L[i]};
      		 		
      		 		\node (C = 1) [fill=pink, circle, left =0.47cm of {l[i + 1] > l[i]}, align=center] {$C = 1$};
      		 		
      		 		\node (l[i + 1] < l[i]) [fill= purple, text= white, diamond, draw = red, below= 0.3cm of {l[i + 1] > l[i]}] {L[i + 1] < L[i]};
      		 		
      		 		\node (D = 1) [fill=pink, circle, left =0.47cm of {l[i + 1] < l[i]}, align=center] {$D = 1$};
      		 		
      		 			
      		 		\node (after if) [fill=black, text= white, rectangle, left= 0.3 of {C = 1}] {AFTER IF};
      		 		
      		 		\node (incremento) [fill=pink, circle, left=1.5cm of while, align=center] {$i = i + 1$};
      		 		
      		 		\node (no ordenada) [fill= purple, text=white, diamond, draw = red, left= 1.34cm of {end}] {$C \land D$};
      		 		
      		 		
      		 		
      		 		\draw[->,thick] (idle) -- (start);
      		 		\draw[->,thick] (start) -- (asignar);
      		 		\draw[->,thick] (asignar) -- (while);
      		 		\draw[->,thick] (start) -- (2,-1.5) -- (2,0) -- (idle);
      		 		\draw[->,thick] (while) -- (i < n - 1);
      		 		\draw[->,thick] (i < n - 1) -- (end);
      		 		\draw[->,thick] (end) -- (-1.83,0) -- (idle);
      		 		\draw[->,thick] (i < n - 1) -- (l[i + 1] > l[i]);
      		 		\draw[->,thick] (l[i + 1] > l[i]) -- (l[i + 1] < l[i]);
      		 		\draw[->,thick] (incremento) -- (while);
      		 	    \draw[->,thick] (l[i + 1] > l[i]) -- (C = 1);
      		 	    \draw[->,thick] (l[i + 1] < l[i]) -- (D = 1);
      		 		\draw[->,thick] (C = 1) -- (after if);
      		 		\draw[->,thick] (l[i + 1] < l[i]) -- (0, -15.5) -- (-5.2, -15.5) -- (-5.2, -10.12);
      		 		\draw[->,thick] (D = 1) -- (-4, -13.23) -- (-4, -10.12);
      		 		\draw[->,thick] (after if) -- (no ordenada);
      		 		\draw[->,thick] (no ordenada) -- (end);
      		 		\draw[->,thick] (no ordenada) -- (-4.53, -5.4) -- (incremento);
      		 		
      		 		
      		 	\end{tikzpicture}
      		 \end{figure}
      		 
      		    
      		 \newpage
      		 \section*{Ejercicio \textcolor{red}{67}}
      		  \rule{\linewidth}{2pt}	
      		 \subsection*{Descripción}
      		 
      		Dada una lista $L$ devolver la lista $L'$ que se obtiene al ordenar sus  elementos, utilizando un método de ordenación $O(n)$. Se conoce que todos los elementos de la lista están en un intervalo desde $a$ a $b$ (ambos enteros); es decir, si el intervalo es de $3$ a $6$, la lista contiene a $3, 4, 5$ y $6$, pero no necesariamente ordenados. Note que el tamaño de la lista depende siempre del intervalo.\newline
      		
      		Dado una lista $L$, de tamaño $n$, que contiene todos los elementos de un intervalo desde algún $a$ a un $b$ (ambos desconocidos a priori), ordene la lista $L$.
      		 
      		 \subsection*{Salida}
      		 
      		 Para la salida debe imprimir la lista $L$ ordenada de menor a mayor, separando cada elemento por un espacio en blanco.\newline
      		 
      		 Por ejemplo: para $L = [4, 3, 5, 6]$ debería imprimir:
      		 
      		 \begin{lstlisting}
      		 	3 4 5 6
      		 \end{lstlisting}
      		 
      		 	\subsection*{Logisim}
      		 
      		 Se dispondrá en INPUT los datos de entrada a partir de la dirección 0. La entrada se estructura de la siguiente forma:
      		 
      		 \begin{itemize}
      		 	
      		 	\item $w_{0}: n $(Tamaño de la lista $L$)
      		 	\item $w_{1:n} : L$
      		 \end{itemize}
      		 
      		 	\subsection*{SASM}
      		 
      		 En la sección .data se deben definir los valores de entrada de la siguiente forma:
      		 
      		 \begin{itemize}
      		 	
      		 	\item $n$ : un número de tamaño $dd$ que representa al tamaño de la lista $L$
      		 	
      		 	\item $array$ : un array de números de tamaño $dd$ que representa $L$
      		 	
      		 	
      		 \end{itemize}
      		 
      		 Por ejemplo, un posible encabezado podría ser:
      		 
      		 \begin{lstlisting}
      		 	section .data
      		 	n dd 4
      		 	array dd 4, 3, 5, 6
      		 \end{lstlisting}
      		 
      		 \subsection*{Seudocódigo}
      		 
      		 \begin{lstlisting} [language=Python]
      		 def Order_On(n, L):
      		 
      			a = 0
      		 	i = 1
      		 	
      		 	while i < n:
      		 	
      		 		if L[i] < L[a]:
      		 			a = i
      		 		
      		 		i = i + 1
      		 
      		 	i = 1
      		 	L[0] = L[a]
      		 
      		 	while i < n :
      		 		L[i] = L[i - 1] + 1
      		 		i = i + 1
      		 		
      		 	return L
      		 \end{lstlisting}
      		  \begin{figure}[ht]
      		 	\centering
      		 	\subsection*{DIAGRAMA}
      		 	\begin{tikzpicture}
      		 		
      		 		\node (idle) [fill= black, text= white, rectangle] {IDLE};
      		 		\node (start) [fill= red, diamond, draw = orange, below= 0.3cm of idle] {START};
      		 		\node (asignar) [fill=gray, text = white, circle, below=0.3cm of start, align=center] {$n = w_0$\\$a = 0$ \\ $i = 1$ \\ $min = L[0]$};
      		 		\node (while) [fill=black, text= white, rectangle, below= 0.3 of asignar] {$WHILE_1$};
      		 		
      		 		\node (i < n) [fill= red, diamond, draw = orange, below= 0.3cm of while] {i < n};
      		 		
      		 		\node (after while) [fill= black, text= white, rectangle, left = 1cm of i < n] {AFTER WHILE};
      		 		
      		 		\node (l[i] < l[a]) [fill= red, diamond, draw = orange, below= 0.3cm of i < n] {L[i] < L[a]};
      		 		
      		 		\node (update) [fill=gray, text = white, circle, below =0.3cm of {l[i] < l[a]}, align=center] {$a = i$};
      		 		
      		 		\node (after if) [fill=black, text= white, rectangle, right= 0.3 of i < n] {AFTER IF};
      		 		
      		 		\node (incremento) [fill=gray, text = white, circle, right=0.32cm of while, align=center] {$i = i + 1$};
      		 		
      		 		\node (reasignacion) [fill=gray, text = white, circle, above=0.47cm of {after while}, align=center] {$i = 1$ \\ L[0] = min};
      		 		
      		 		
      		 		\node (while_2) [fill=black, text= white, rectangle, above= 0.3 of reasignacion] {$WHILE_2$};
      		 	
      		 		\node (i < n_0) [fill= red, diamond, draw = orange, above= 1cm of while_2] {i < n};
      		 		
      		 		\node (ordenar) [fill=gray, text = white, circle, left=0.32cm of {i < n_0}, align=center] { L[i] = L[i - 1] + 1\\$i = i + 1$};
      		 		
      		 		\draw[->,thick] (idle) -- (start);
      		 		\draw[->,thick] (start) -- (asignar);
      		 		\draw[->,thick] (asignar) -- (while);
      		 		\draw[->,thick] (start) -- (2,-1.5) -- (2,0) -- (idle);
      		 		\draw[->,thick] (while) -- (i < n);
      		 		\draw[->,thick] (i < n) -- (after while);
      		 		\draw[->,thick] (i < n) -- (l[i] < l[a]);
      		 		\draw[->,thick] (l[i] < l[a]) -- (update);
      		 		\draw[->,thick] (incremento) -- (while);
      		 		\draw[->,thick] (after if) -- (incremento);
      		 		\draw[->,thick] (l[i] < l[a]) -- (1.5, -9.49) -- (1.5, -7.43);
      		 		\draw[->,thick] (update) -- (2.6,-11.57) -- (2.6, - 7.43);
      		 		\draw[->,thick] (after while) -- (reasignacion);
      		 		\draw[->,thick] (reasignacion) -- (while_2);
      		 		\draw[->,thick] (while_2) -- (i < n_0);
      		 		\draw[->,thick] (i < n_0) -- (-1.5, -1.75) -- (-1.5, 0) -- (idle);
      		 		\draw[->,thick] (i < n_0) -- (ordenar);
      		 		\draw[->,thick] (ordenar) -- (-5.8, -3.8) -- (while_2);
      		 		
      		 	\end{tikzpicture}
      		 \end{figure}
      		 \newpage
      		 
      		\section*{Ejercicio \textcolor{red}{89}}
      				 \rule{\linewidth}{2pt}	
      	    \subsection*{Descripción}
      	    
      	    Dado un radio $r$ de una circunferencia hallar el área de la misma.
      	     
      	     \subsection*{Salida}
      	     
      	     Para la salida debe imprimir el valor del área de la circunferencia en $cm^2$. Asuma que $PI = 3, PI$ es un entero.\newline
      	     
      	     Por ejemplo: para $r = 2$ debería imprimir:
      	     
      	     \begin{lstlisting}
      	     	12
      	     \end{lstlisting}
      	     
      	     	\subsection*{Logisim}
      	     
      	     Se dispondrá en INPUT los datos de entrada a partir de la dirección 0. La entrada se estructura de la siguiente forma:
      	     
      	     \begin{itemize}
      	     	
      	     	\item $w_{0}: r$
      	     \end{itemize}
      	     
      	     \subsection*{SASM}
      	     
      	     En la sección .data se deben definir los valores de entrada de la siguiente forma:
      	     
      	     \begin{itemize}
      	     	
      	     	\item $r$ : un número de tamaño $dw$ que representa $r$
      	     	
      	     \end{itemize}
      	     
      	     Por ejemplo, un posible encabezado podría ser:
      	     
      	     \begin{lstlisting}
      	     	section .data
      	     	r dw 2
      	     \end{lstlisting}
      			
      			
      			 \subsection*{Seudocódigo}
      			
      			\begin{lstlisting} [language=Python]
      			def Area (r):
      				return 3*r*r
      			\end{lstlisting}
      		   
      		   \begin{figure}[ht]
      		   	\centering
      		   	 \subsection*{DIAGRAMA}
      		   	\begin{tikzpicture}
      		   	
      		   		\node (idle) [fill= magenta, text= white, rectangle] {IDLE};
      		   		\node (start) [fill= teal, diamond, draw = orange, below= 0.3cm of idle] {START};
      		   		\node (asignar) [fill=lime, circle, draw=yellow, below= 0.3cm of start] {$A = 3 \cdot r^2$};
      		   		\node (end) [fill=magenta, text= white, rectangle, below= 0.3 of asignar] {END};
      		   	    
      		   	
      		   		\draw[->,thick] (idle) -- (start);
      		   		\draw[->,thick] (start) -- (asignar);
      		   		\draw[->,thick] (asignar) -- (end);
      		   		\draw[->,thick] (end) -- (-2, -5.08) -- (-2,0) -- (idle);
      		   		\draw[->,thick] (start) -- (2,-1.5) -- (2,0) -- (idle);
      		   	
      		   	\end{tikzpicture}
      		   \end{figure}
      		   
      		 
      		   
      		   \begin{table}[ht]
      		   	\centering
      		   	  \subsection*{ASIGNACIONES}
      		   	\begin{tabular}{|c|c|c|}
      		   		\hline
      		   		$L_{a}$ & $r$ & $A$ \\
      		   		\hline
      		   		0 & $r$ & 0 \\
      		   		\hline
      		   		1 & $r$ & $3\cdot r^2$ \\
      		   		\hline
      		   	\end{tabular}
      		   
      		   \end{table}
      		   
      		   
      		   
	 \end{flushleft}
	 
	 
\end{document}