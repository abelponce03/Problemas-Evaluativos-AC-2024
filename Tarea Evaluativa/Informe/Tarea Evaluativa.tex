%%%%%%%%%%%%%%%%%%%%%%%%%%%%%%%%%%%%%%%%%%%%%%%%%%%%%%%%%%%%%%%%%%%%%%%%%%%
%
% Plantilla para un artículo en LaTeX en español.
%
%%%%%%%%%%%%%%%%%%%%%%%%%%%%%%%%%%%%%%%%%%%%%%%%%%%%%%%%%%%%%%%%%%%%%%%%%%%

% Qué tipo de documento estamos por comenzar:
\documentclass[a4paper]{article}
% Esto es para que el LaTeX sepa que el texto está en español:
\usepackage[spanish]{babel}
\selectlanguage{spanish}
% Esto es para poder escribir acentos directamente:
\usepackage[utf8]{inputenc}
\usepackage[T1]{fontenc}
\usepackage{listings}
\usepackage{amsmath}
\usepackage{graphicx}
\usepackage{color}
\usepackage{xcolor}
\usepackage[utf8]{inputenc}
\usepackage[T1]{fontenc}
\usepackage{listings}
\usepackage{tikz}
\usepackage{listingsutf8}
\usetikzlibrary{shapes,arrows,positioning}


\lstset{
	language=[x86masm]Assembler,
	inputencoding=utf8,
	language=Python,
	basicstyle=\ttfamily\small,
	commentstyle=\color{green!60!black},
	keywordstyle=\color{blue},
	stringstyle=\color{red},
	numbers=left,
	numberstyle=\tiny\color{gray},
	stepnumber=1,
	numbersep=5pt,
	backgroundcolor=\color{white},
	showspaces=false,
	showstringspaces=false,
	showtabs=false,
	frame=single,
	rulecolor=\color{black},
	tabsize=4,
	captionpos=b,
	breaklines=true,
	breakatwhitespace=false,
	title=\lstname,
	escapeinside={\%*}{*)},
	morekeywords={def,for,in,range,if,else,return},
	}


%% Asigna un tamaño a la hoja y los márgenes
\usepackage[a4paper,top=3cm,bottom=2cm,left=3cm,right=3cm,marginparwidth=1.75cm]{geometry}

%% Paquetes de la AMS
\usepackage{amsmath, amsthm, amsfonts}
%% Para añadir archivos con extensión pdf, jpg, png or tif
\usepackage{graphicx}
\usepackage[colorinlistoftodos]{todonotes}
\usepackage[colorlinks=true, allcolors=blue]{hyperref}

%% Primero escribimos el título
\title{Tarea Evaluativa }
\author{Abel Ponce González\\
	Richard Alejandro Matos Arderí	
}


%% Después del "preámbulo", podemos empezar el documento

\begin{document}
	\maketitle
	
	\begin{flushleft}
	
      		\section{Ejercicio 18}
      		
      		
      		\subsection*{Descripción}
      		
      		Dada una lista $L$ de $n$ elementos $a_{1}, a_{2}, ... a_{n}$ devolver el índice del mínimo elemento de $L$.
      		
      		\subsection*{Salida}
      		
      		Para la salida debe imprimir el mínimo elemento de $L$.
      		
      		Por ejemplo: para $L = [4, 3, 5, 6]$ debería imprimir: 
      		
      		\begin{lstlisting}
      			1
      		\end{lstlisting}      		
      		
      		\subsection*{Logisim}
      		
      		Se dispondrá en INPUT los datos de entrada a partir de la dirección 0. La entrada se estructura de la siguiente forma:
      		
      		\begin{itemize}
      			
      			\item $w_{0}: n $(Tamaño de la lista $L$)
      			\item $w_{1:n} : L$
      		\end{itemize}
      		
      		\subsection*{SASM}
      		
      		En la sección .data se deben definir los valores de entrada de la siguiente forma:
      		
      		\begin{itemize}
      			
      			\item $n$ : un número de tamaño $dd$ que representa al tamaño de la lista $L$
      			
      			\item $array$ : un array de números de tamaño $dd$ que representa $L$
      			
      			
      		\end{itemize}
      	
      		Por ejemplo, un posible encabezado podría ser:
      		
      		\begin{lstlisting}
      			section .data
      			n dd 4
      			array dd 4, 3, 5, 6
      		\end{lstlisting}
      		
      		\subsection*{Código Ensamblador}
      		
      		Este código recorre la lista y mantiene un registro del mínimo encontrado junto con su índice. Al final, el registro ebx contendrá el índice del elemento mínimo.
      		
      		\section{Ejercicio 62}
      		
      		\subsection*{Descripción}
      		
      		 Determinar si una lista $a_{1}, a_{2}, ... a_{i}$ está ordenada en orden creciente o decreciente.
      		 
      		 \subsection*{Salida}
      		 
      		 Para la salida debe imprimir la C si está ordenada en orden creciente o D lo está en orden decreciente.\newline
      		 
      		 Por ejemplo: para $L = [3, 4, 6]$ debería imprimir:
      		 
      		 \begin{lstlisting}
      		 	C
      		 \end{lstlisting}
      		 
      		\subsection*{Logisim}
      		
      		Se dispondrá en INPUT los datos de entrada a partir de la dirección 0. La entrada se estructura de la siguiente forma:
      		
      		\begin{itemize}
      			
      			\item $w_{0}: n $(Tamaño de la lista $L$)
      			\item $w_{1:n} : L$
      		\end{itemize}
      		 
      		 	\subsection*{SASM}
      		 
      		 En la sección .data se deben definir los valores de entrada de la siguiente forma:
      		 
      		 \begin{itemize}
      		 	
      		 	\item $n$ : un número de tamaño $dd$ que representa al tamaño de la lista $L$
      		 	
      		 	\item $array$ : un array de números de tamaño $dd$ que representa $L$
      		 	
      		 	
      		 \end{itemize}
      		 
      		 Por ejemplo, un posible encabezado podría ser:
      		 
      		 \begin{lstlisting}
      		 	section .data
      		 	n dd 3
      		 	array dd 3, 4, 6
      		 \end{lstlisting}
      		 
      		 
      		 \section{Ejercicio 67}
      		 
      		 \subsection*{Descripción}
      		 
      		Dada una lista $L$ devolver la lista $L'$ que se obtiene al ordenar sus  elementos, utilizando un método de ordenación $O(n)$. Se conoce que todos los elementos de la lista están en un intervalo desde $a$ a $b$ (ambos enteros); es decir, si el intervalo es de $3$ a $6$, la lista contiene a $3, 4, 5$ y $6$, pero no necesariamente ordenados. Note que el tamaño de la lista depende siempre del intervalo.\newline
      		
      		Dado una lista $L$, de tamaño $n$, que contiene todos los elementos de un intervalo desde algún $a$ a un $b$ (ambos desconocidos a priori), ordene la lista $L$.
      		 
      		 \subsection*{Salida}
      		 
      		 Para la salida debe imprimir la lista $L$ ordenada de menor a mayor, separando cada elemento por un espacio en blanco.\newline
      		 
      		 Por ejemplo: para $L = [4, 3, 5, 6]$ debería imprimir:
      		 
      		 \begin{lstlisting}
      		 	3 4 5 6
      		 \end{lstlisting}
      		 
      		 	\subsection*{Logisim}
      		 
      		 Se dispondrá en INPUT los datos de entrada a partir de la dirección 0. La entrada se estructura de la siguiente forma:
      		 
      		 \begin{itemize}
      		 	
      		 	\item $w_{0}: n $(Tamaño de la lista $L$)
      		 	\item $w_{1:n} : L$
      		 \end{itemize}
      		 
      		 	\subsection*{SASM}
      		 
      		 En la sección .data se deben definir los valores de entrada de la siguiente forma:
      		 
      		 \begin{itemize}
      		 	
      		 	\item $n$ : un número de tamaño $dd$ que representa al tamaño de la lista $L$
      		 	
      		 	\item $array$ : un array de números de tamaño $dd$ que representa $L$
      		 	
      		 	
      		 \end{itemize}
      		 
      		 Por ejemplo, un posible encabezado podría ser:
      		 
      		 \begin{lstlisting}
      		 	section .data
      		 	n dd 4
      		 	array dd 4, 3, 5, 6
      		 \end{lstlisting}
      		 
      		 
      		 \section{Ejercicio 89}
      			
      	    \subsection*{Descripción}
      	    
      	    Dado un radio $r$ de una circunferencia hallar el área de la misma.
      	     
      	     \subsection*{Salida}
      	     
      	     Para la salida debe imprimir el valor del área de la circunferencia en $cm^2$. Asuma que $PI = 3, PI$ es un entero.\newline
      	     
      	     Por ejemplo: para $r = 2$ debería imprimir:
      	     
      	     \begin{lstlisting}
      	     	12
      	     \end{lstlisting}
      	     
      	     	\subsection*{Logisim}
      	     
      	     Se dispondrá en INPUT los datos de entrada a partir de la dirección 0. La entrada se estructura de la siguiente forma:
      	     
      	     \begin{itemize}
      	     	
      	     	\item $w_{0}: r$
      	     \end{itemize}
      	     
      	     \subsection*{SASM}
      	     
      	     En la sección .data se deben definir los valores de entrada de la siguiente forma:
      	     
      	     \begin{itemize}
      	     	
      	     	\item $r$ : un número de tamaño $dw$ que representa $r$
      	     	
      	     \end{itemize}
      	     
      	     Por ejemplo, un posible encabezado podría ser:
      	     
      	     \begin{lstlisting}
      	     	section .data
      	     	r dw 2
      	     \end{lstlisting}
      			
	 \end{flushleft}
	 
	 
\end{document}